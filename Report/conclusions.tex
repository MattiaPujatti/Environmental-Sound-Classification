% !TEX root = report.tex

\section{Concluding Remarks}
\label{sec:conclusions}

The goal of this project was to study and implement the main approaches developed so far in the field of environmental sound classification, especially considering the limited nature of dataset available for our purposes. So, given the ESC-50 dataset, we have presented the main techniques designed and implemented with the objective of classifying those clips. Even if we didn't managed to reach the very high accuracies of several professional works, we can be quite satisfied of several results obtained and summarized in the last paragraph of the previous section.\\
Further improvements could be simply deepening one of the models analyzed, maybe trying with more complex architectures, different preprocessing steps or even testing pre-trained models. However, actually the biggest problem is the absence of large datasets, since 2000 clips is not enough to construct a credible statistics for generic applications, even exploiting nested cross validation. \\
In any case, the strong interest of the scientific community in speech recognition tasks, and the consequent improvements in sound classification, will for sure benefit also the branch of environmental sounds. \\
Regarding myself, instead, I can say that I practiced a lot with several Python libraries dedicated to machine learning (\textit{SKLearn, Keras, Tensorflow}), especially for what involves data management and preprocessing. Moreover, this was my first project exploiting sound data, and so I learned a lot about audio features and track manipulation.\\  
The most challenging part of this project was, for sure, determining a way to speed up the training processes (that could take up to many hours for each model), keeping, at the same time, a low memory usage, and ending up with a reliable statistics for the results. 




